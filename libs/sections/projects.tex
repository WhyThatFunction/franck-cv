% ===== Practical / Academic Projects =====
\section*{Ausgewählte Labor- \& Studienprojekte}

\resumeentry{Okt 2019 -- Jul 2021}{FAU \textbar\ Grundlagen der Elektrotechnik (EMF / HF)}{Laborprojekt}
\begin{cvbullets}
  \item Schaltungsaufbau \& Messungen: Oszilloskop, Funktionsgenerator; Frequenzgang, Fehlerbilder systematisch prüfen.
  \item Analogtechnik: OPV/Brückenschaltungen, nichtlineare Bauteile; Dimensionierung, Toleranzen, Abgleich.
  \item EMF, HF-Grundlagen: Induktion, magnetische Felder, saubere Messprotokolle und Auswertung.
\end{cvbullets}
\entryspace

\resumeentry{Okt 2022 -- Mär 2023}{FAU \textbar\ Grundlagen der elektrischen Antriebstechnik}{Laborprojekt}
\begin{cvbullets}
  \item Antriebstechnik: DC-Maschinenanalyse, U/f-Steuerung Asynchronmaschine; Kennlinien, Wirkzusammenhänge.
  \item Versuche \& Safety: Messaufbau, Datenaufnahme, Auswertung (z.\,B. Drehzahl/Strom/Spannung), Protokolle.
\end{cvbullets}
\entryspace

\resumeentry{Apr 2023 -- Jul 2023}{FAU \textbar\ Mechatronische Systeme}{Teamprojekt}
\begin{cvbullets}
  \item Systementwicklung: Anforderungen, mechanische/elektrische Auslegung, Integration von Sensorik/Aktorik.
  \item Inbetriebnahme: Testplan, Funktionschecks, Debugging, iterative Optimierung; Teamabstimmung.
\end{cvbullets}
