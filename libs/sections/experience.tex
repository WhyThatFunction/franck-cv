% ===== Professional Experience & Education =====
\section*{Berufserfahrung \& Ausbildung}

\roleheader{Bachelorand \textbar\ Werkstofftechnik / SLS}{Bayerisches Polymer Institut \textbar\ Fürth (Bayern)}{Jan 2025 -- Jul 2025}
\begin{cvbullets}
  \item Verarbeitung polymerer Pulver für SLS-Anlagen und Durchführung additiver Fertigungsprozesse.
  \item Werkstoffanalysen mittels LOI, UL94, Cone Calorimetrie und TGA; strukturierte Auswertung der Ergebnisse.
  \item Ableitung von Prozessoptimierungen auf Basis experimenteller Daten.
  \item Erstellung technischer Berichte sowie Präsentation der Ergebnisse für Forschung \& Entwicklung.
\end{cvbullets}

\roleheader{Werkstudent Mechatronik}{Graphite Materials GmbH \textbar\ Oberasbach, Bayern}{Mai 2024 -- Dez 2024}
\begin{cvbullets}
  \item Analyse und Optimierung von Steuerungs-/Automatisierungssystemen industrieller Anlagen.
  \item Recherche, Auswahl und Integration geeigneter Norm- und Zukaufteile.
  \item Durchführung von Systemtests sowie strukturierte Fehleranalyse.
  \item Erstellung und Pflege technischer Dokumentationen; Zusammenarbeit mit interdisziplinären Teams.
\end{cvbullets}

\textbf{Bachelor of Science (B.Sc.) Mechatronik} \dateright{Apr 2019 -- Sep 2025}\\
{\color{Muted}Friedrich-Alexander-Universität Erlangen–Nürnberg (FAU)}\\
\begin{cvbullets}
  \item Schwerpunkt Elektronik und Schaltungstechnik mit Laborpraxis in EMF, Hochfrequenztechnik und Antriebstechnik.
  \item Verknüpfung der Studieninhalte mit Werkstudenten- und Bachelorarbeitserfahrungen (SLS, Prüfstandsbau, Fehleranalyse).
  \item Dokumentation und Präsentation technischer Ergebnisse in deutsch- und englischsprachigen Teams.
\end{cvbullets}
