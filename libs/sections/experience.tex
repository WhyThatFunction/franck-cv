% ===== Professional Experience =====
\section*{Berufserfahrung}

\resumeentry{Jan 2025 -- Jul 2025}{Bayerisches Polymer Institut \textbar\ Fürth (Bayern)}{Bachelorand \textbar\ Werkstofftechnik / SLS}
\begin{cvbullets}
  \item SLS-Prozessfenster: Pulverhandling, Probenaufbau, Parameter-Iterationen; Dokumentation in Versuchsreihen.
  \item Werkstoffprüfung (LOI, UL94, TGA, Cone): Probenpräparation, Messmittelcheck, Validierung.
  \item Datenauswertung (Excel/MATLAB): Trends, Vergleichsmetriken, Ableitung konkreter Prozessoptimierungen.
  \item Abstimmung mit F\&E: Findings präsentieren, Entscheidungen begründen, technische Reports DE/EN.
\end{cvbullets}
\entryspace

\resumeentry{Mai 2024 -- Dez 2024}{Graphite Materials GmbH \textbar\ Oberasbach, Bayern}{Werkstudent Mechatronik}
\begin{cvbullets}
  \item Systemtests \& Fehlersuche: I/O-Signale prüfen, Sensor-/Aktorik verifizieren, Ursachenanalyse bis Abstellmaßnahme.
  \item Automatisierung: Funktionschecks an Steuerung, Peripherie; einfache Ablauf-/Interlock-Logik nachvollziehen.
  \item Konstruktion/Teileauswahl: Norm- \& Zukaufteile bewerten, Einbausituation/CAD prüfen, Stücklistenpflege.
  \item Zusammenarbeit: Schnittstellen mit Fertigung/F\&E, saubere Dokumentation (Prüfprotokolle, Änderungen).
\end{cvbullets}
