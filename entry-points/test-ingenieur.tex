% Franck Kengne Tobou — LaTeX CV Template
% Structure inspired by the first CVs: clear header + summary + experience + education + skills + projects/practical + languages
% Compile with: xelatex or lualatex (recommended for modern fonts)

\documentclass[11pt,a4paper]{article}

% ---------- Packages ----------
\usepackage[margin=1.6cm]{geometry}
\usepackage{enumitem}
\usepackage{hyperref}
\usepackage{xcolor}
\usepackage{tabularx}

\usepackage{titlesec}
\usepackage{parskip}

% ---------- Style ----------
\setlength{\parindent}{0pt}
\definecolor{Accent}{HTML}{1F4E79}
\definecolor{Muted}{HTML}{555555}

\hypersetup{
  colorlinks=true,
  urlcolor=Accent,
  linkcolor=Accent
}

\titleformat{\section}{\large\bfseries\color{Accent}}{}{0pt}{}[\titlerule]
\titlespacing*{\section}{0pt}{10pt}{6pt}

\newcommand{\cvtitle}[1]{{\Huge\bfseries #1}}
\newcommand{\cvsubtitle}[1]{{\large\color{Muted} #1}}

\newcommand{\cvitem}[2]{\textbf{#1}\\#2}

\newcommand{\dateright}[1]{\hfill{\color{Muted}#1}}

\newcommand{\roleheader}[3]{% role, company/location, date
  \textbf{#1} \dateright{#3}\\
  {\color{Muted}#2}\\
}

\newenvironment{cvbullets}{
  \begin{itemize}[leftmargin=*, itemsep=2pt, topsep=2pt]
}{
  \end{itemize}
}

% ---------- Document ----------
\begin{document}

% ===== Header =====
\begin{minipage}[t]{0.68\textwidth}
  \cvtitle{FRANCK GIRES KENGNE TOBOU}\\
  \cvsubtitle{Mechatronikingenieur (B.Sc.) \;|\; Quality \& Test \;|\; R\&D \;|\; Manufacturing \/ Materials}
\end{minipage}
\begin{minipage}[t]{0.31\textwidth}
  \raggedleft
  Rothenburger Str. 148\\
  90439 Nürnberg\\
  \href{mailto:franckkengne.fk@gmail.com}{franckkengne.fk@gmail.com}\\
  +49 176 7631 0195\\
  % \href{https://www.linkedin.com/in/yourprofile}{LinkedIn: yourprofile}
\end{minipage}

\vspace{6pt}

% ===== Summary / Profile =====
\section*{Profil}
Motivierter Mechatronikingenieur (B.Sc.) mit praktischer Erfahrung in industrieller Qualitätssicherung, Werkstofftechnik und mechatronischen Systemen. Fundierte Kenntnisse in Mess- und Prüftechnik, additiver Fertigung (SLS) sowie Fehleranalyse und Prozessoptimierung. Teamorientiert, analytisch und sicher im Arbeiten in deutsch- und englischsprachigen technischen Umgebungen.

% ===== Focus Areas / Core Competencies (inspired by first CV “Focus Areas”) =====
\section*{Fokusbereiche}
\begin{tabularx}{\textwidth}{@{}X X@{}}
\textbf{Qualität \& Test} & \textbf{Mechatronische Systeme}\\
Mess- und Prüftechnik, Systemtests, Fehleranalyse, Dokumentation & Funktionsanalyse, Auslegung von Komponenten, Integration\\
\\
\textbf{Werkstoffe \& Fertigung} & \textbf{Daten \& Tools}\\
SLS, Pulververarbeitung, LOI/UL94/Cone Calorimetrie/TGA, Prozessoptimierung & Auswertung, Berichte/Präsentationen, MS Office, MATLAB\\
\end{tabularx}

% ===== Professional Experience =====
\section*{Berufserfahrung}

\roleheader{Bachelorand \textbar\ Werkstofftechnik / SLS}{Bayerisches Polymer Institut \textbar\ Fürth (Bayern)}{Jan 2025 -- Jul 2025}
\begin{cvbullets}
  \item Verarbeitung polymerer Pulver für SLS-Anlagen und Durchführung additiver Fertigungsprozesse.
  \item Werkstoffanalysen mittels LOI, UL94, Cone Calorimetrie und TGA; strukturierte Auswertung der Ergebnisse.
  \item Ableitung von Prozessoptimierungen auf Basis experimenteller Daten.
  \item Erstellung technischer Berichte sowie Präsentation der Ergebnisse für Forschung \& Entwicklung.
\end{cvbullets}

\roleheader{Werkstudent Mechatronik}{Graphite Materials GmbH \textbar\ Oberasbach, Bayern}{Mai 2024 -- Dez 2024}
\begin{cvbullets}
  \item Analyse und Optimierung von Steuerungs-/Automatisierungssystemen industrieller Anlagen.
  \item Recherche, Auswahl und Integration geeigneter Norm- und Zukaufteile.
  \item Durchführung von Systemtests sowie strukturierte Fehleranalyse.
  \item Erstellung und Pflege technischer Dokumentationen; Zusammenarbeit mit interdisziplinären Teams.
\end{cvbullets}

% ===== Education =====
\section*{Ausbildung}
\textbf{Bachelor of Science (B.Sc.) Mechatronik} \dateright{Apr 2019 -- Sep 2025}\\
{\color{Muted}Friedrich-Alexander-Universität Erlangen–Nürnberg (FAU)}\\
Schwerpunkt: Elektronik und Schaltungstechnik

% ===== Practical / Academic Projects (mirrors “Projects” section idea) =====
\section*{Ausgewählte Labor- \& Studienprojekte}
\textbf{Grundlagen der Elektrotechnik (EMF / Hochfrequenztechnik, FAU)} \dateright{Okt 2019 -- Jul 2021}\\
\begin{cvbullets}
  \item Arbeit mit Oszilloskop, Funktionsgeneratoren und Labormessgeräten; Durchführung und Auswertung von Messreihen.
  \item Schaltungen mit nichtlinearen Bauteilen, Brückenschaltungen und Operationsverstärkern.
  \item Induktionsmessungen und Untersuchung magnetischer Felder.
\end{cvbullets}

\textbf{Grundlagen der elektrischen Antriebstechnik (FAU)} \dateright{Okt 2022 -- März 2023}\\
\begin{cvbullets}
  \item Analyse von Gleichstrommaschinen; U/f-Steuerung einer Asynchronmaschine.
  \item Versuchsdurchführung, Messdatenanalyse und Dokumentation.
\end{cvbullets}

\textbf{Mechatronische Systeme (FAU)} \dateright{Apr 2023 -- Jul 2023}\\
\begin{cvbullets}
  \item Entwicklung eines mechatronischen Systems inkl. Auslegung elektrischer und mechanischer Komponenten.
  \item Tests, Funktionsanalyse, Fehlerdiagnose und Optimierung von Hardwarekomponenten.
\end{cvbullets}

% ===== Skills (structured like first CV “Skills” blocks) =====
\section*{Kompetenzen}
\begin{tabularx}{\textwidth}{@{}l X@{}}
\textbf{Mess- \& Prüftechnik} & Oszilloskop, Multimeter, Labormessgeräte, Aufbau von Versuchständen \\
\textbf{Software \& Tools} & PTC Creo, Arduino, MATLAB, MS Office \\
\textbf{Programmierung} & C, C\#, Java, CNC (Grundkenntnisse) \\
\textbf{Arbeitsweise} & Analytisch, selbstständig, sorgfältig, hohes technisches Verständnis, lern- und leistungsbereit, teamorientiert \\
\end{tabularx}

% ===== Languages =====
\section*{Sprachkenntnisse}
\begin{tabularx}{\textwidth}{@{}l X@{}}
\textbf{Französisch} & fließend \\
\textbf{Englisch} & fließend \\
\textbf{Deutsch} & fließend \\
\end{tabularx}

% ===== Other =====
\section*{Sonstiges}
Führerschein Klasse B

% ===== Optional: Certifications =====
% \section*{Zertifikate}
% (Nur hinzufügen, wenn vorhanden oder in Arbeit mit geplantem Abschlussdatum)

\end{document}
